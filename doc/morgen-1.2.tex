\documentclass[%
aspectratio=43 % the default if ommited
%aspectratio=169 %switches the page layout to 16:9 format
,table % required for colored tables (passes the table option to xcolor)
]{beamer}


% [short title] the short title of the presentation is ignored by the new theme.
\title{\texttt{morgen} (1.2)}
% subtitle
\subtitle{Model Order Reduction for Gas and Energy Networks}
% [short author] author of the presentation
% short author is shown in the footline on each frame
\author{C.~Himpe, S.~Grundel} 
% institute
\institute{CSC}
% date
\date{}
%\date{November 5, 2010}


% beamertheme with options
\mode<presentation>
{
  \usetheme[%
  ,english            % change the language of logos and footline to english,
                      % recommended
  % The following add the corresponding group logo to the header. Note that the
  % implementation does not check for multiple selections, but the standard only
  % expects a single group to appear.
  % ,ARB
  % ,BPE
   ,CSC
  % ,DRI
  % ,MSD
  % ,NDS
  % ,PCF
  % ,PSD
  % ,PSE
  % ,SCT
  % Choice of the title page layout:
   ,layoutone             % Layout with fancy graphics and color gradient
  % ,layouttwo            % Layout with blueish box (the default)
  % ,layoutthree          % Layout with fancy title graphics of choice
                          % use \pgfdeclareimage{title}{} to define it as below.
  %,groupnameinhead        % Adds the full group name to the logo in the header
  %,compacthead           % moves the subtitle below the frametitle in the head
                          % instead of in the slide area
  %,adjustframetitle      % automatically fit oversized frametitles horizontally
  %,hideaffiliationheader % hides the word Affiliations: or Zugehörigkeiten: on
                          % the title page
  ,hidefundingheader     % hides the words "suppoted by" or "gefördert durch"
                          % above the funding logos 
   ,serifmath                % switch to serif math if desired
  % ,ovgu                 % adds the OVGU Logo to the header if no group logo is
                          % used. 
  % ,noauthor             % switches off the display of the author's name in the
                          % footline 
  , titleinfoot           % show the short title in the footline
  , movepagenums          % make the footline more symmetric in layout
  % , hidepagenums        % do not print page numbers in footline
  % ,navsymbols           % beamer navigation symbols are off by default and can
                          % be activated using this option
  % ,fullscreen            % try to display the PDF in fullscreen mode by default
   ,singlepage            % try to display teh PDF in single page (rather than
                          % continuous view) mode by default
  % ,alertbold            % show alerted text in bold face (this option will
                          % make \alert cause errors in math mode)
  ]{mpi2015}            
}


% Definition of title page logo boxes. Both commands support an optional
% argument for a fill/background color in case the logos get unreadable due to
% transparency. 
%
% When defining your own logos make sure they are
% .075\paperwidth high, e.g. by using 
%  \includegraphics[height=.075\paperheight]{mylogo}


% Partnerlogos for the titlepage: (comment for "none")
%\cooplogos[white]{\ovgu}
%\cooplogos{\ovgu}

% Funding Agency and Project Cluster Logos (comment for "none")
\fundinglogos{% Add spaces when using multiple logos
  % \bmbf        % Federal Ministry for Education and Research
  % \bmwi        % Federal Ministry for Economic Affairs and Energy
  % \daad        % Deutscher Akademischer Austauschdienst
  %  \dfg         % Deutsche Forschungsgemeinschaft
  % \ecost       % European COST action
  % \horizon     % Horizon 2020 programme
  % \imprsproeng % IMPRS Production Engineering Magdeburg
}

% Define the graphics used on the title page in layout 1 & 3:
\pgfdeclareimage[width=.6\linewidth,height=.45\linewidth]{title}{MPI_outside.jpg}

\usepackage{stackengine}
\usepackage{graphbox}
\newcommand{\me}{\includegraphics[align=c,height=1.15em]{mathenergy.pdf}~}

\begin{document}
% title page
\maketitle % use maketitle to make the title frame

%%%%%%%%%%%%%%%%%%%%%%%%%%%%%%%%%%%%%%%%%%%%%%%%%%%%%%%%%%%%%%%%%%%%%%%%%%%%%%%%

\begin{frame}\frametitle{Get \texttt{morgen}}

\begin{center}\Huge
 \alert{\url{https://git.io/morgen}}
\end{center}

\end{frame}

%%%%%%%%%%%%%%%%%%%%%%%%%%%%%%%%%%%%%%%%%%%%%%%%%%%%%%%%%%%%%%%%%%%%%%%%%%%%%%%%

\begin{frame}\frametitle{The Name \texttt{morgen}}

\LARGE

\texttt{morgen} {\large(german for: tomorrow)}
\setlength{\leftmargini}{3.5em}
\begin{itemize}

 \item \textbf{M}odel

 \item \textbf{O}rder

 \item \textbf{R}eduction

 \item[for] \textbf{G}as

 \item[and] \textbf{E}nergy

 \item \textbf{N}etworks

\end{itemize}

\end{frame}

%%%%%%%%%%%%%%%%%%%%%%%%%%%%%%%%%%%%%%%%%%%%%%%%%%%%%%%%%%%%%%%%%%%%%%%%%%%%%%%%

\begin{frame}\frametitle{Why \texttt{morgen}}

\Large

\begin{itemize}

 \item Testing model-solver-reductor combinations

 \item Comparing models, solvers, or reductors

 \item Benchmarking reductors

 \item Prototyping algorithms

 \item Uncertainty quantification

\end{itemize}

\end{frame}

%%%%%%%%%%%%%%%%%%%%%%%%%%%%%%%%%%%%%%%%%%%%%%%%%%%%%%%%%%%%%%%%%%%%%%%%%%%%%%%%

\begin{frame}\frametitle{About \texttt{morgen}}

\Large

\begin{itemize}

 \item Open-source {\normalsize (BSD-2-Clause)}

 \item High-Level {\normalsize (MATLAB and OCTAVE)}

 \item Modular {\normalsize (Six modules)}

 \item Configurable {\normalsize (Three-level configuration)}

 \item Extensible {\normalsize (Contributions welcome)}

\end{itemize}

\end{frame}

%%%%%%%%%%%%%%%%%%%%%%%%%%%%%%%%%%%%%%%%%%%%%%%%%%%%%%%%%%%%%%%%%%%%%%%%%%%%%%%%

\begin{frame}\frametitle{\texttt{morgen} Modules}

\Large

\begin{enumerate}

 \item Models {\normalsize (Discretizations)}

 \item Solvers {\normalsize (Integrators)}

 \item Reductors {\normalsize (Model Reduction Algorithms)}

 \item Networks {\normalsize (Topologies \& Scenarios)}

 \item Tests {\normalsize (Simulation and Model Reduction Experiments)}

 \item Tools {\normalsize (Unit and Format Converters)}

\end{enumerate}

\end{frame}

%%%%%%%%%%%%%%%%%%%%%%%%%%%%%%%%%%%%%%%%%%%%%%%%%%%%%%%%%%%%%%%%%%%%%%%%%%%%%%%%

\begin{frame}\frametitle{Models}

\large

\textbf{Implemented:}
\begin{itemize}\itemsep1ex

 \item \texttt{ode\_mid} -- midpoint discretization

 \item \texttt{ode\_end} -- endpoint discretization (port-Hamiltonian)

\end{itemize}

\begin{center}

 \texttt{discrete = model(network,config);}

\end{center}

\normalsize

\texttt{discrete} \textbf{Structure:}
\vskip-1.2em
\begin{columns}

\begin{column}[t]{.45\textwidth}

\begin{itemize}\itemsep1ex

 \item \texttt{.E} -- Mass matrix function

 \item \texttt{.A} -- System matrix

 \item \texttt{.B} -- Input matrix

 \item \texttt{.C} -- Output matrix

 \item \texttt{.f} -- Nonlinear vector field

\end{itemize}

\end{column}

\begin{column}[t]{.55\textwidth}

\begin{itemize}\itemsep1ex

 \item \texttt{.J} -- Vector field Jacobian

 \item \texttt{.x0} -- Initial state

 \item \texttt{.nP} -- Number of pressure states

 \item \texttt{.nQ} -- Number of mass-flux states

 \item \texttt{.nPorts} -- Number of ports

\end{itemize}

\end{column}

\end{columns}

\end{frame}

%%%%%%%%%%%%%%%%%%%%%%%%%%%%%%%%%%%%%%%%%%%%%%%%%%%%%%%%%%%%%%%%%%%%%%%%%%%%%%%%

\begin{frame}\frametitle{Solvers}

\vskip1ex

\textbf{Implemented:}
\begin{itemize}\itemsep.5ex\small

 \item \texttt{imex1} -- 1st Order Implicit-Explicit (Euler-Euler)

 \item \texttt{imex2} -- 2nd Order Implicit-Explicit (Runge-Kutta)

 \item \texttt{cnab2} -- 2nd Order Crank-Nicolson/Adams-Bashforth

 \item \texttt{generic} -- 2nd Order Adaptive Rosenbrock (\texttt{ode23s})

 \item \texttt{rk4} -- ``Classic'' 4th Order Explicit Runge-Kutta

 \item \texttt{rk2hyp} -- 2nd Order Explicit Runge-Kutta with increased stability

 \item \texttt{rk4hyp} -- 4th Order Explicit Runge-Kutta with increased stability

\end{itemize}

\begin{center}

 \texttt{solution = solver(discrete,scenario,config);}

\end{center}

\normalsize

\texttt{solution} \textbf{Structure:}
\vskip-1.2em
\begin{columns}

\begin{column}[t]{.43\textwidth}
\begin{itemize}\itemsep1ex\small

 \item \texttt{.t} -- Time instances

 \item \texttt{.u} -- Input time series

 \item \texttt{.y} -- Output time series

 \item \texttt{.runtime} -- Solver runtime

\end{itemize}
\end{column}

\begin{column}[t]{.58\textwidth}
\begin{itemize}\itemsep1ex\small

 \item \texttt{steady\_z0} -- Mean compressibility

 \item \texttt{steady\_error} -- Steady-state error

 \item \texttt{steady\_iter1} -- Algebraic Iterations

 \item \texttt{steady\_iter2} -- Differential Iterations 

\end{itemize}
\end{column}

\end{columns}

\end{frame}

%%%%%%%%%%%%%%%%%%%%%%%%%%%%%%%%%%%%%%%%%%%%%%%%%%%%%%%%%%%%%%%%%%%%%%%%%%%%%%%%

\begin{frame}\frametitle{Reductors}

\vskip1ex

\large

\textbf{Implemented:}
\begin{itemize}\itemsep1ex\small

 \item \texttt{pod\_r} -- Structured Proper Orthogonal Decomposition (POD)

 \item {\footnotesize $\begin{smallmatrix} \text{\texttt{eds\_ro}}, & \text{\texttt{eds\_wx}}, & \text{\texttt{eds\_wz}}, \\
                                           \text{\texttt{eds\_ro\_l}}, & \text{\texttt{eds\_wx\_l}}, & \text{\texttt{eds\_wz\_l}} \end{smallmatrix}$} -- Structured Dominant Subspaces

 \item {\footnotesize $\begin{smallmatrix} \text{\texttt{mpod\_ro}}, & \text{\texttt{mpod\_wx}}, & \text{\texttt{mpod\_wz}}, \\
                                           \text{\texttt{mpod\_ro\_l}}, & \text{\texttt{mpod\_wx\_l}}, & \text{\texttt{mpod\_wz\_l}} \end{smallmatrix}$} -- Structured Modified POD

 \item {\footnotesize $\begin{smallmatrix} \text{\texttt{bpod\_ro}}, \\ \text{\texttt{bpod\_ro\_l}} \end{smallmatrix}$}  -- Structured Balanced POD

 \item {\footnotesize $\begin{smallmatrix} \text{\texttt{ebt\_ro}}, & \text{\texttt{ebt\_wx}}, & \text{\texttt{ebt\_wz}}, \\ 
                                           \text{\texttt{ebt\_ro\_l}}, & \text{\texttt{ebt\_wx\_l}}, & \text{\texttt{ebt\_wz\_l}} \end{smallmatrix}$} -- Structured Balanced Truncation

 \item \texttt{gopod\_r} -- Structured Goal-Oriented POD

 \item {\footnotesize $\begin{smallmatrix} \text{\texttt{ebg\_ro}}, & \text{\texttt{ebg\_wx}}, & \text{\texttt{ebg\_wz}}, \\
                                           \text{\texttt{ebg\_ro\_l}}, & \text{\texttt{ebg\_wx\_l}}, & \text{\texttt{ebg\_wz\_l}} \end{smallmatrix}$} -- Structured Balanced Gains

 \item \texttt{dmd\_r} -- Structured Dynamic Mode Decomposition Galerkin

\end{itemize}

\begin{center}

 \scalebox{.95}{\texttt{[proj,name] = reductor(solver,discrete,scenario,config);}}

\end{center}

\begin{itemize}\itemsep.3ex

 \item \texttt{proj} -- Cell array of projectors

 \item \texttt{name} -- Full name of reductor

\end{itemize}

\end{frame}

%%%%%%%%%%%%%%%%%%%%%%%%%%%%%%%%%%%%%%%%%%%%%%%%%%%%%%%%%%%%%%%%%%%%%%%%%%%%%%%%

\begin{frame}\frametitle{Networks \& Tests}

\textbf{Network and scenario data:}
\begin{itemize}

 \item Network data stored as decorated edge list in CSV format (\texttt{.net}). 

 \item Scenario data stored as key-vale pairs in INI format (\texttt{.ini}).

 \item Network base name determines associated scenario folder name.

 \item Each network has minimally a \texttt{training.ini} scenario.

\end{itemize}

~\\

\textbf{Types of tests:}
\begin{itemize}

 \item Prefix \texttt{sim\_} -- Simulate scenario by a model-solver combination.

 \item Prefix \texttt{mor\_} -- Reduce and test model-solver-reductor combination.

\end{itemize}

\end{frame}

%%%%%%%%%%%%%%%%%%%%%%%%%%%%%%%%%%%%%%%%%%%%%%%%%%%%%%%%%%%%%%%%%%%%%%%%%%%%%%%%

\begin{frame}\frametitle{Tools}

\large

\textbf{Available:}
\begin{itemize}

 \item \texttt{xml2net} -- Convert \emph{GasLib} \texttt{.xml} to \texttt{morgen} \texttt{.net}

 \item \texttt{json2net} -- Convert \emph{MathEnergy} \texttt{.json} to \texttt{morgen} \texttt{.net}

 \item \texttt{csv2net} -- Convert \emph{SciGRID\_gas} \texttt{.csv} to \texttt{morgen} \texttt{.net}

 \item \texttt{vf2kgs} -- Convert volume flow to mass flow in kg/s

 \item \texttt{psi2bar} -- Convert pressure from psi to bar

 \item \texttt{cmp\_friction} -- Compare friction factors

 \item \texttt{cmp\_compressibility} -- Compare compressibility factors

 \item \texttt{randscen} -- Generate random scenario from training scenario

\end{itemize}

\end{frame}

%%%%%%%%%%%%%%%%%%%%%%%%%%%%%%%%%%%%%%%%%%%%%%%%%%%%%%%%%%%%%%%%%%%%%%%%%%%%%%%%

\begin{frame}\frametitle{Configuration}

\large

\textbf{Available:}
\begin{itemize}

 \item optional arguments (\texttt{varargin})

 \item configuration file (\texttt{morgen.ini})

 \item fallback via hard-coded default values

\end{itemize}

\end{frame}

%%%%%%%%%%%%%%%%%%%%%%%%%%%%%%%%%%%%%%%%%%%%%%%%%%%%%%%%%%%%%%%%%%%%%%%%%%%%%%%%

\begin{frame}\frametitle{Use \texttt{morgen}}

\begin{center}
 \scalebox{0.81}{\texttt{R = morgen(network\_id,scenario\_id,model\_id,solver\_id,reductor\_ids,varargin);}}
\end{center}

\setlength{\leftmargini}{5em}
\begin{itemize}

 \item[\{string\}] \texttt{network\_id} -- Network file (\texttt{.net}) base name

 \item[\{string\}] \texttt{scenario\_id} -- Scenario file (\texttt{.ini}) base name

 \item[\{string\}] \texttt{model\_id} -- Model function name

 \item[\{string\}] \texttt{solver\_id} -- Solver function name

 \item[\{cell\}] \texttt{reductor\_ids} -- Array of reductor names

 \item[\{string\}] \texttt{varargin} -- Adhoc configuration arguments (\texttt{`key=val`})

\end{itemize}

\end{frame}

%%%%%%%%%%%%%%%%%%%%%%%%%%%%%%%%%%%%%%%%%%%%%%%%%%%%%%%%%%%%%%%%%%%%%%%%%%%%%%%%

\begin{frame}\frametitle{Notes}

\large

~\\

\begin{itemize}

 \item \texttt{morgen} is open source {\normalsize (under BSD-2-Clause license)},

 \item and compatible with MATLAB and Octave.

 \item A template model, solver and reductor are available.

 \item Currently, all reductors use \texttt{emgr}: \alert{\url{https://gramian.de}}.

 \item See \alert{\href{https://github.com/mpimd-csc/morgen/blob/main/README.md}{\texttt{README.md}}} for more info.

 \item This is research software!

\end{itemize}

\end{frame}

%%%%%%%%%%%%%%%%%%%%%%%%%%%%%%%%%%%%%%%%%%%%%%%%%%%%%%%%%%%%%%%%%%%%%%%%%%%%%%%%

\begin{frame}\frametitle{Summary}

\large

\begin{center}\large
\texttt{morgen} -- \textbf{M}odel \textbf{O}rder \textbf{R}eduction for \textbf{G}as and \textbf{E}nergy \textbf{N}etworks \\
\vskip11ex\LARGE
 $\rightarrow$~~\alert{\url{https://git.io/morgen}}~~$\leftarrow$
\end{center}

~\\~\\~\\~\\

C.~Himpe, S.~Grundel, P.~Benner: \\
\textbf{Model Order Reduction for Gas and Energy Networks}. \\
Journal of Mathematics in Industry 11: 13, 2021. \\
\alert{\url{https://doi.org/10.1186/s13362-021-00109-4}}

\end{frame}

%%%%%%%%%%%%%%%%%%%%%%%%%%%%%%%%%%%%%%%%%%%%%%%%%%%%%%%%%%%%%%%%%%%%%%%%%%%%%%%%

\begin{frame}\frametitle{References}

\scriptsize

\begin{itemize}

 \item C.~Himpe, S.~Grundel: \textbf{System Order Reduction for Gas and Energy Networks}; Proceedings in Applied Mathematics and Mechanics: ?--?, 2022. \href{https://doi.org/}{\texttt{doi:}}

 \item C.~Himpe, S.~Grundel, P.~Benner: \textbf{Next-Gen Gas Network Simulation}; Progress in Industrial Mathematics at ECMI 2021: ?--?, 2022. \href{https://doi.org/}{\texttt{doi:}}

 \item C.~Himpe, S.~Grundel, P.~Benner: \textbf{Efficient Gas Network Simulations}; in: German Success Stories in Industrial Mathematics: 17--22, 2022. \href{https://doi.org/10.1007/978-3-030-81455-7\_4}{\texttt{doi:10.1007/978-3-030-81455-7\_4}}

 \item C.~Himpe, S.~Grundel, P.~Benner: \textbf{Model Order Reduction for Gas and Energy Networks}; Journal of Mathematics in Industry 11: 13, 2021. \href{https://doi.org/10.1186/s13362-021-00109-4}{\texttt{doi:10.1186/s13362-021-00109-4}}

 \item T.~Clees, A.~Baldin, P.~Benner, S.~Grundel, C.~Himpe, B.~Klaassen, F.~Küsters, N.~Marheineke, L.~Nikitina, I.~Nikitin, J.~Pade, N.~Stahl, C.~Strohm, C.~Tischendorf, A.~Wirsen: \textbf{MathEnergy – Mathematical Key Technologies for Evolving Energy Grids}; in: Mathematical Modeling, Simulation and Optimization for Power Engineering and Management: 233--262, 2021. \href{https://doi.org/10.1007/978-3-030-62732-4\_11}{\texttt{doi:10.1007/978-3-030-62732-4\_11}} 

 \item P.~Benner, S.~Grundel, C.~Himpe, C.~Huck, T.~Streubel, C.~Tischendorf: \textbf{Gas Network Benchmark Models}; in: Applications of Differential-Algebraic Equations: Examples and Benchmarks: 171--197, 2019. \href{https://doi.org/10.1007/11221\_2018\_5}{\texttt{doi:10.1007/11221\_2018\_5}}

 \item P.~Benner, S.~Grundel, C.~Himpe: \textbf{Parametric Model Order Reduction for Gas Flow Models}; ScienceOpen Posters MoRePas 4, 2018. \href{https://doi.org/10.14293/P2199-8442.1.SOP-MATH.EJOCET.v1}{\texttt{doi:10.14293/P2199-8442.1.SOP-MATH.EJOCET.v1}}

\end{itemize}

\end{frame}

\end{document}

